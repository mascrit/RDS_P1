\documentclass[letterpaper]{article}

\usepackage[spanish]{babel}
\usepackage{graphicx}
\usepackage{hyperref}
\usepackage{subfiles}
\usepackage{fancyhdr}
\usepackage{float}
\usepackage[left=3cm,right=3cm,top=3cm,bottom=4cm]{geometry}

\pagestyle{fancy}
\fancyhead[L]{\thepage}
\fancyfoot[C]{\includegraphics[width=0.1\textwidth]{inge_logo}}


\graphicspath{{img/}}

\begin{document}

\begin{titlepage}
  \centering
  \includegraphics[width=0.5\textwidth]{unam_logo}\vfill{}
  {\scshape\Huge Facultad de Ingeniera\par}\vspace{0.5cm}
  {\scshape\Large Redes de Datos Seguras\par}\vfill
  {\huge \textbf{Proyecto 1}\\Planeación, optimización y
    rediseño de una red cableada}\vfill
  
  {\Large
    Alumnos\begin{itemize}
    \item Garrido Czacki Mario Horacio
    \item Romero Andrade Cristian
    \item Romero Andrade Vicente

    \end{itemize}
  }\vfill
  {\large Profesor: Ing.~Edgar Martinez Meza}\vfill
  \includegraphics[width=0.1\textwidth]{inge_logo}
  
  
\end{titlepage}

\tableofcontents{}\newpage

\section{Descripción}\label{sec:desc}

Elaborar la planeación, optimización y rediseño de la red Cableada interna del
Instituto de Geografía de la UNAM.\ El diseño de la red abarcará aspectos físicos
y lógicos (cableado estructurado y direccionamiento lógico), así como la
aplicación de los conceptos estudiados en los tema 3 y 5 de la materia de Redes
de Datos Seguras.

\subsection{Escenario}\label{sec:esc}

La red que se implementará abarca el edificio Principal del Instituto del Instituto
de Geografía. Es necesario tener las siguientes consideraciones:
\begin{itemize}
\item El enlace de acometida principal deberá ser con tecnología de fibra
  óptica y se tomará desde el anillo de red UNAM, nota éste ya existe.
  
\item  En el edificio Principal existen dos Terrazas en la que no se puede realizar
el cableado, sin embargo se necesita conectividad.

\item También existen áreas donde no se puede realizar cableado pero se
  necesita conectividad. (Checar en los planos)
  
\item  Los cuartos de telecomunicaciones el MDF y los IDF’s sólo pueden
instalarse en áreas permitidas, éstos deben estar conectados a través de
fibra óptica, entre cada uno de los IDF’s y el MDF.\@

\item Los cubículos son ocupados por un investigador y sus becarios y las áreas
más grandes llamadas peceras albergan varios becarios. Considere el
número de nodos adecuado para cada área y las direcciones IP que se
van a requerir.

\item  En caso de que haya más de un área de trabajo por piso deberá aplicar
direccionamiento lógico VLSM y poner las IP ́s correspondientes a cada
área.
\end{itemize}


\section{Análisis}\label{sec:ana}

\subfile{contenido/analisis}




\end{document}
