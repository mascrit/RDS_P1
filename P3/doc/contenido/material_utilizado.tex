\documentclass[../main.tex]{subfiles}

\begin{document}
\subsection{Red Emulada}\label{sec:red_emu}

\begin{table}[H]
  \centering
  \begin{tabular}{|rl|}
    \hline
    Segmento:&$192.168.100.0/24$\\\hline
    Puerta de enlace:&$192.168.100.1$\\\hline
    Broadcast:&$192.168.100.255$\\\hline
    Dominio:&\lstinline|srv.nis|\\\hline
  \end{tabular}
\end{table}

\subsection{Servidor}\label{sec:servidor}

\subsubsection{Servidor Linux (VM)}\label{sec:slvm}


\begin{table}[H]
  \centering
  \begin{tabular}{|rl|}
    \hline
    Hostname: &Node03\\\hline{}
    Sistema Operativo: & Debian 10 \textit{Buster}\\\hline
  \end{tabular}
\end{table}

\subsubsection{Tarjeta de Red}\label{sec:tr}

\begin{table}[H]
  \centering
  \begin{tabular}{|rl|}
    \hline{}
    IP:&$192.16.100.119/24$\\\hline{}
    DNS:&$192.168.100.119 8.8.8.8$\\\hline
  \end{tabular}
\end{table}

\subsection{Configuración}\label{sec:serv_conf}

\subsubsection{Configurar la tarjeta de red}\label{sec:conf_tr}

Se tiene que configurar la tarjeta de red para que adquiera su DNS y
ip estática:

\begin{itemize}
\item En este caso la interfaz de red es \lstinline|ens33|, donde
  este nombre puede variar.
\item Se tiene que modificar el archivo \lstinline|/etc/network/interfaces|
  y añadir la siguiente configuración:

  \begin{lstlisting}
auto ens33
allow-hotplug ens33
iface ens33 inet static
    address 192.168.100.119
    netmask 255.255.255.0
    network 192.168.100.0
    broadcast 102.168.100.255
    gateway 192.168.100.1
    dns-nameservers 192.168.100.119 8.8.8.8
    dns-search srv.nis
  \end{lstlisting}
\end{itemize}

\paragraph{Asignar Dominio}\ \\
Se debe de añadir la siguiente línea a \lstinline|/etc/hosts|.

\begin{lstlisting}
192.168.100.119 Node03.srv.nis srv.nis Node03 srv
\end{lstlisting}

Esto redirecciona todas las peticiones del dominio del servidor a su ip. El
gestor de DNS configura de forma automática el registro en
\lstinline|/etc/resolv.conf|, quedando de la siguiente manera:

\begin{lstlisting}
# Dynamic resolv.conf(5) file for glibc resolver(3) generated by resolvconf(8)
#     DO NOT EDIT THIS FILE BY HAND -- YOUR CHANGES WILL BE OVERWRITTEN
nameserver 192.168.100.119
nameserver 8.8.8.8
search srv.nis
\end{lstlisting}

\subsubsection{NIS}\label{sec:nis}

NIS funciona para poder centralizar la autenticación de los clientes Linux.

\begin{enumerate}
\item Instalar NIS, en terminal con permisos administrativos:

  \begin{lstlisting}[language=bash]
apt -y install nis
  \end{lstlisting}

  Al finalizar aparecerá una pantalla de configuración donde se
  añadirá el dominio del servidor

  \begin{lstlisting}
NIS domain:

srv.nis______

    <ok>
\end{lstlisting}
  
\item Configurar como servidor maestro NIS

  Se tiene que modificar el archivo \lstinline|/etc/default/nis|

  \begin{lstlisting}
# Linea 6: Poner a NIS como servidor maestro
NISSERVER=master
\end{lstlisting}

  Adicionalmente en el mismo archivo de configuración, se puede
  configurar un rango de IPs que pueden hacer peticiones
  a este servicio

  \begin{lstlisting}
# Si se deja asi se le dara acceso a todo el mundo
0.0.0.0 0.0.0.0
# Si se configura asi se le dara acceso solo al rango deseado
192.168.100.0 192.168.100.255
\end{lstlisting}

  Reiniciamos el servicio nis para que se efectúen los cambios.

  \begin{lstlisting}[language=bash]
systemctl restart nis
\end{lstlisting}

  
\item Aplicar la configuración al servicio

  Ejecutamos el siguiente comando

  \begin{lstlisting}
/usr/lib/yp/ypinit -m
\end{lstlisting}

  Si todo va bien se tiene que aparecer lo siguiente:

  \begin{lstlisting}
Node03.srv.nis has been set up as a NIS master server.

Now you can run ypinit -s Node03.srv.nis on all slave server.
\end{lstlisting}
  
\item Cada que se tenga  que añadir un nuevo usuario se
  tiene que actualizar la base de datos de NIS\@
  (este ya esta incluido en el script \lstinline|add_user.sh|).

  Se ejecuta el siguiente comando dentro del directorio
  \lstinline|/var/yp|

  \begin{lstlisting}
make
\end{lstlisting}
  
\end{enumerate}

\subsubsection{NFS}\label{sec:nfs}

NFS crea un sistema de archivos centralizados por redefined
\begin{enumerate}
\item Instalar el servidor nfs

  \begin{lstlisting}[language=bash]
apt -y install nfs-kernel-server
\end{lstlisting}

\item Configurar el dominio del servidor en el
  archivo \lstinline|/etc/idmapd.conf|

  \begin{lstlisting}
# Linea 6: Aqui se descomenta y se agrega el dominio
Domain = srv.nis
\end{lstlisting}

\item Añadir la ruta de los directorios home que se van a
  compartir por NFS, esto es en el archivo \lstinline|/etc/exports|

  \begin{lstlisting}
/home 192.168.100.0/24(rw,no_root_squash,no_subtree_check) 
\end{lstlisting}

  \begin{itemize}
  \item \lstinline|/home| es la ruta donde se van a montar
    los directorios personales de los clientes.
  \item \lstinline|xx.xx.xx.xx/xx| Es la mascara del segmento que
    puede acceder a estos directorios por NFS.\@
  \item \lstinline|(..*)| Son las opciones de exports.
  \end{itemize}

  
\item Reiniciar el servicio para ver reflejados los cambios.

  \begin{lstlisting}
systemctl restart nfs-server
\end{lstlisting}
  
\end{enumerate}



\end{document}
